\documentclass[12pt]{article}
\usepackage[utf8]{inputenc}
\usepackage{hyperref}
\usepackage{indentfirst}
\topmargin=-0.45in
\evensidemargin=0in
\oddsidemargin=0in
\textwidth=6.5in
\textheight=9.0in
\headsep=0.25in

\title{Rapport : Projet Programmation 1}
\author{Léo Juguet}
\date{}


\begin{document}
\maketitle
\newpage

\section{Le langage}
L'implémentation des entiers s'est fait sans problème.
La gestion des flottants a demandé un peu plus de temps
affin de comprendre le fonctionnement de la pile.

Le code assembleur généré n'est pas un code optimisé. Ainsi il 
y a souvent des éléments empilé dans la pile puis dépilé juste après.
Il serait possible de remplacer ces 2 instructions par un move.
\subsection{Implémentation}
Grâce à l'utilisation de ocamllex et ocamlyacc dès la première minute,
le lexer et le parser ont été très rapide a faire. De plus l'ajout
ou la modification de fonctionnalités ont pu se faire très facilement.

Le typage de l'arbre syntaxique se fait lors du parsing, grâce à un 
arbre syntaxique bien typé, sauf pour un cas (voir dans la section Implémentation des variables).
La traduction en code assembleur se fait ensuite avec un simple parcours de l'arbre. Chaque opération
dépile le nombre d'éléments dont elle a besoin puis empile le résultat sur la pile. S'il n'y a pas de variables,
à la fin on dépile le dernier éléments puis on appelle printf avec cet éléments. Grâce au typage de notre arbre syntaxique,
il est facile de déterminer si l'on doit afficher un flottants ou un entier.

\section{Bonus}
\subsection{Opérateurs supplémentaire}

La factorielle a été implémenté pour les entiers positifs. La factorielle pour des entiers négatifs donne un résultat non prévue.

La puissance pour des entiers ou des flottants par un entier positif est implémenté. La puissance par des entiers négatifs donne un résultat non prévue.

L'implémentation de ces deux fonctions utilise un appel à une fonctionne qui est ajouté au code assembleur si la fonction est appelé.

La division entre float est également implémenté avec le symbole : \verb|/.|

\subsection{Gestion de variables}
Les variables sont implémentées. La syntaxe est la suivante : 

\begin{verbatim}
NomDeVariable = expression arithmétique
...
NomDeVariable = expression arithmétique
expression arithmétique

\end{verbatim}

Il est important de noter que la dernière expression arithmétique est obligatoire, pour que le fichier \verb|.exp|
puisse être compilé. Seul le résultat de la dernière expression arithmétique est affiché lors de l'exécution
du programme résultant de la compilation du fichier assembleur donné par le compilateur aritha.

Il est possible d'utiliser une variable dans une expression arithmétique, à condition que celle-ci soit définis
dans une instruction précédente et qu'elle soit du bon type.

Il est également possible de changer la valeur d'une variable avec la même syntaxe que pour l'assignation. Cela modifie également le
type de la variable.

\subsubsection{Implémentation des variables}
Avec l'implémentation des expressions arithmétiques vue précédemment, on peut montrer qu'à la fin de l'évaluation d'une 
expression arithmétique, il n'y a qu'un seul résultat empilé dans la pile et ce résultat est le résultat de l'expression arithmétique.
\textit{(modulo des possible overflow dû à l'encodage des entiers et flottants)}.
Donc pour affecter une valeur à une variables, il suffit d'évaluer l'expression arithmétique, puis de sauvegarder cette variables à un 
emplacement bien choisi.

Une première approche a été de sauvegarder les variables à la base de la pile, et d'avoir un compteur qui compte "de combien on est loin" du 
"bloc des variables" 
\footnote{en incrémentant le compteur si on ajoute des éléments sur la pile et en le décrémenant si on en enlève. 
Ainsi l'addition décrémente de 1 le compteur, alors que la déclaration d'un entier l'incrémente de 1, et la factorielle ne change pas sa valeur.}, 
puis d'y accéder en ajoutant ce compteur à la position de la tête de la pile (\verb|rsp|). Si cette méthode fonctionne, pour
éviter des sources d'erreurs potentielle, la facon d'accéder aux variables a légèrement changé.

Dans la seconde approche, on sauvegarde toujours les variables à la base de la pile. Cepandant au lieu d'y accéder depuis le registre 
\verb|rsp|, on met la valeur de \verb|rsp| dans \verb|rpb| au début du programme. Ainsi dans \verb|rpb| on a la position de la base de la pile. Il 
suffit alors de savoir dans quel ordre ont été sauvegardées les variables pour y accéder.
Pour cela dans notre assembler a chaque déclaration de variables on stocke le nom de la variable et son type dans une file.
(Le type de la variable est bien définis lors des déclarations car l'arbre syntaxique des expressions arithmétiques est bien typé 
lors du parsing \footnote{Sauf pour un cas qui est traité très prochainement}).

Si l'on veux redéfinir une variable, cela est possible. Ainsi le code suivant renverra \verb|10|
\begin{verbatim}
x = 5
x = 5 + x
x
\end{verbatim}

Pour faire cela, on vérifie simplement si la variable a déjà été définis (est dans la file des variables).
Si c'est le cas, alors on évalue l'expression arithmétique de la redéfinition, puis on modifie la valeur de la variable dans la pile.
Cette opération est toujours bien typé.


Une seul opération n'est pas bien typé à la sortie du parser. Cette opération correspond à si il n'y a qu'une variable dans l'expression 
arithmétique. Ainsi la dernière ligne de l'exemple ci-dessus, n'est pas bien typé à la sortie du parser. Pour résoudre ce problème,
on parcours simplement la file de nos variable. Si la variable a été définis précédement alors la variable se trouve dans la file,
et son type aussi, on connait alors le type de la variable et donc de l'expression arithmétique.


\end{document}
